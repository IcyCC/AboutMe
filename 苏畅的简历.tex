% !TEX TS-program = xelatex
% !TEX encoding = UTF-8 Unicode
% !Mode:: "TeX:UTF-8"

\documentclass{resume}
\usepackage{zh_CN-Adobefonts_external} % Simplified Chinese Support using external fonts (./fonts/zh_CN-Adobe/)
% \usepackage{NotoSansSC_external}
% \usepackage{NotoSerifCJKsc_external}
% \usepackage{zh_CN-Adobefonts_internal} % Simplified Chinese Support using system fonts
\usepackage{linespacing_fix} % disable extra space before next section
\usepackage{cite}

\begin{document}
\pagenumbering{gobble} % suppress displaying page number

\name{苏畅}

\basicInfo{
  \email{sam.suchang@qq.com} \textperiodcentered\ 
  \phone{(+86) 153-1107-0339} \textperiodcentered\ 
  \faGithub{https://github.com/IcyCC}\textperiodcentered\ 
  \faEye{C++/Go后台开发}
  }

 \section{\faGraduationCap\  教育背景}
\datedsubsection{\textbf{北京林业大学}, 北京}{2016 -- 至今}
\textit{在读本科生}\ \quad  网络工程专业, 预计 2020 年 9 月毕业 

\section{\faUsers\ 实习经历}
\datedsubsection{\textbf{知乎}}{2018.06-2018.09}
\role{实习后端工程师}{社区事业部-内容}
\textbf{工作内容} \quad 负责ops平台社区内容标注平台的前后端, 和 ops平台接入工具的开发

\datedsubsection{\textbf{木瓜移动}}{2017.07-2017.10}
\role{实习后端工程师}{木瓜广告}
\textbf{工作内容} \quad 负责广告采集脚本的优化, 广告审核后台的开发优化, DSP 供应方平台接口接入

\section{\faCogs\ 个人技能}
% increase linespacing [parsep=0.5ex]
\begin{itemize}[parsep=0.5ex]
    \item 丰富的 Python 开发经验, 作为主力语言开发多个商业项目, 并对其虚拟机有了解, 有其 \textbf{C/C++ 混合}开发经验
    \item 较熟悉 C++, 了解 \textbf{C11} 值语义, 内存管理, 并发等新特性并在个人项目中实践. 熟悉Golang
    \item 拥有\textbf{全栈开发}和架构设计能力,熟悉Python和Go的 web 开发和vue前端, 主导开发过\textbf{5+}个较大商业项目
    \item 有并发网络编程经验, 熟悉 Reactor模型, 了解muduo, asyncio等网络库的设计并在 unix 下实现过网络库
    \item 熟悉 MySQL, 了解 Redis 的设计原理
    \item 熟悉 常用数据结构及算法, 熟悉常见网络安全的知识, 曾自己编写的加密算法实现一个简单的 SSL
\end{itemize}

\section{\faBook\ 个人成果}
\datedsubsection{\textbf{Reimu -- 基于 Reactor 模式的多线程 Morden C++ 网络库}}{\faGithub https://github.com/IcyCC/reimu}
\role{C++, Socket, 并发, io复用, Reactor模型}{}
\begin{onehalfspacing}
\textbf{项目介绍}:\c 为学习C++网络编程,所编写的一个 Reactor 模式的并发网络库, 基于 SafeQueue 做多线程的同步, 加入了自己的一些思考
\end{onehalfspacing}

\datedsubsection{\textbf{AsyncEasyapi -- 快速后端增删改查API工具}}{\faGithub https://github.com/IcyCC/async\_easyapi}
\role{Python,Asyncio,Flask,Sqlalchemy, 元编程}{}
\begin{onehalfspacing}
\textbf{项目介绍}:\quad 为了提升重复业务的开发效率, 基于Python的元类编写的效率提升工具, 可通过10 行代码完成一个单表 curd 的 API, 并约定了错误处理, 权限处理等流程的规范
\end{onehalfspacing}

\datedsubsection{\textbf{WhParser -- 一个支持类文华财经语法 dsl 的量化交易框架}}{}
\role{Python\ Cython Ply}{商业项目}
\begin{onehalfspacing}
\textbf{项目介绍}:\quad 一个期货量化回测框架, 接入上期 CTP交易接口, 支持类似文化财经语法的 DSL. 使用 pybind11 和 cython 进行Python/C++ 混合开发
\end{onehalfspacing} 

\datedsubsection{\textbf{知乎文章}}{https://www.zhihu.com/people/su-chang-60-5/posts}
\section{\faObjectGroup 近期团队项目}
\datedsubsection{\textbf{北京林业大学督导系统}}{\textbf{项目负责人} \  团队规模 3-6人}
\role{Python Vue Redis Kafka Mysql Mongodb, Gevent}{商业项目}
\begin{onehalfspacing}
  \textbf{项目介绍}:\quad 为解决老版本系统难以维护的问题, 基于老版本在前后端分离和\textbf{模块解耦}方面做了升级, 设计并增加了\textbf{动态问卷}和\textbf{教学质量报告渲染}的功能, 方便了之后的人进行维护, 现在该系统已负责全校的督导教评业务
\end{onehalfspacing}

\datedsubsection{\textbf{傲禾智能测土配肥系统 }}{\textbf{项目负责人} \  团队规模 3人 }
\role{Golang, Gin, ELectron, Vue, libffi}{商业项目}
\begin{onehalfspacing}
  \textbf{项目介绍}:\quad 用于测土配肥的程序, golang做服务器后端, 根据测土的结果计算复合肥配比, 客户端通过ffi调用dll 控制配肥流程,  并进行客户,消费,土地,加盟店等信息的管理, 现已经在全国近百家加盟店中投入使用.
\end{onehalfspacing}
\end{document}
