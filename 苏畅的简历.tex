% !TEX TS-program = xelatex
% !TEX encoding = UTF-8 Unicode
% !Mode:: "TeX:UTF-8"

\documentclass{resume}
\usepackage{zh_CN-Adobefonts_external} % Simplified Chinese Support using external fonts (./fonts/zh_CN-Adobe/)
% \usepackage{NotoSansSC_external}
% \usepackage{NotoSerifCJKsc_external}
% \usepackage{zh_CN-Adobefonts_internal} % Simplified Chinese Support using system fonts
\usepackage{linespacing_fix} % disable extra space before next section
\usepackage{cite}

\begin{document}
\pagenumbering{gobble} % suppress displaying page number

\name{苏畅}

\basicInfo{
  \email{sam.suchang@qq.com} \textperiodcentered\ 
  \phone{(+86) 153-1107-0339} \textperiodcentered\ 
  \faGithub{https://github.com/IcyCC}\textperiodcentered\ 
  \faEye{C++/Go后台开发}
  }
  


 \section{\faGraduationCap\  教育背景}
\datedsubsection{\textbf{北京林业大学}, 北京}{2016 -- 至今}
\textit{在读本科生}\ \quad  网络工程专业, 预计 2020 年 9 月毕业 \\
\textit{高分课程}\  \quad  程序设计基础: 97 \quad C++程序设计:92 \quad 	计算机网络: 92 \quad UNIX环境程序设计:87

\section{\faUsers\ 实习经历}
\datedsubsection{\textbf{知乎}}{2018-05-2018.9}
\role{实习后端工程师}{社区事业部-内容}
\textbf{工作内容} \quad 负责ops平台社区内容标注平台的前后端, 和 ops平台接入工具的开发

\datedsubsection{\textbf{木瓜移动}}{2018-05-2018.9}
\role{实习后端工程师}{木瓜广告}
\textbf{工作内容} \quad 负责广告采集脚本的优化, 广告审核后台的开发优化, DSP 供应方平台接口接入

\section{\faCogs\ 个人技能}
% increase linespacing [parsep=0.5ex]
\begin{itemize}[parsep=0.5ex]
  \item 编程语言: 熟练掌握Python, 了解其C/C++拓展开发.较为熟悉 Morden C++, 熟悉 Golang, JavaScript.
  \item Web后端:熟练掌握 MySQL 和 Redis,
  了解 Nginx, Kafka,  Docker
  有良好的架构设计意识
  \item Web前端: 熟练掌握Vue 了解React 拥有\textbf{全栈开发}的能力
  \item 计算机网络: 掌握HTTP协议, 了解TCP/IP协议 , Socket编程 和 常见的并发模型 
  \item 计算机科学常识: 熟练掌握常见的算法以及数据结构,
  了解Unix及其环境编程, 
  了解常见的网络安全知识
\end{itemize}

\section{\faBook\ 个人成果}

\datedsubsection{\textbf{Reimu -- 基于 Reactor 模式的多线程 Morden C++ 网络库}}{\faGithub https://github.com/IcyCC/reimu}
\role{C++\ Socket\ 并发\ io复用}{}
\begin{onehalfspacing}
\textbf{项目介绍}:\c 为学习C++网络编程, 实现了一个基于SafeQueue做线程同步的Poll+EventLoop+线程池的并发网络库, 并设计了Buffer+Slice机制优化了字符串拷贝, 学习到了C++网络编程中
所需要的各种细节知识.
\end{onehalfspacing}

\datedsubsection{\textbf{WhParser -- 一个支持类文华财经语法 dsl 的量化交易框架}}{}
\role{Python\ Cython
 Ply}{商业项目}
\begin{onehalfspacing}
\textbf{项目介绍}:\quad 基于开源回测框架进行修改, 在其基础之上支持了使用类似文华财经语法的DSL进行交易略的编写, 接入了处理后自有数据源使其可以支持
分钟级别和夜盘的期货回测, 接入了CTP接口进行实盘相关开发.
\end{onehalfspacing}


\datedsubsection{\textbf{AsyncEasyapi -- 快速后端增删改查API工具}}{\faGithub https://github.com/IcyCC/async\_easyapi}
\role{Python C++ Cython Ply}{}
\begin{onehalfspacing}
\textbf{项目介绍}:\quad 为了提升平常业务开发的开发效率, 基于Python的元类编写了这个效率提升工具, 可通过继承几个类实现后端Api的快速开发, 已经在多个个人生产项目中采用.
\end{onehalfspacing}

\datedsubsection{\textbf{知乎文章}}{https://www.zhihu.com/people/su-chang-60-5/posts}

\section{\faObjectGroup 团队项目}
\datedsubsection{\textbf{北京林业大学督导系统v2.0}}{\textbf{项目负责人} \  团队规模 3-6人}
\role{Python Vue Redis Kafka Mysql Mongodb, Gevent}{商业项目}
\begin{onehalfspacing}
  \textbf{项目介绍}:\quad 为了解决老版本系统难以维护的问题, 基于老版本系统在前后端分离和\textbf{模块解耦}方面做了升级, 设计并增加了\textbf{动态问卷}和\textbf{教学质量报告渲染}的功能, 方便了之后的人进行维护
\end{onehalfspacing}

\datedsubsection{\textbf{傲禾智能测土配肥系统}}{\textbf{项目负责人} \  团队规模 3人 }
\datedsubsection{\textbf{航空农林喷洒智能监控与面积计量系统}}{\textbf{项目负责人} \  团队规模 4人 }
\datedsubsection{\textbf{微信日历小程序}}{\textbf{项目负责人} \  团队规模 4人 }

\end{document}
