% !TEX TS-program = xelatex
% !TEX encoding = UTF-8 Unicode
% !Mode:: "TeX:UTF-8"

\documentclass{resume}
\usepackage{zh_CN-Adobefonts_external} % Simplified Chinese Support using external fonts (./fonts/zh_CN-Adobe/)
% \usepackage{NotoSansSC_external}
% \usepackage{NotoSerifCJKsc_external}
% \usepackage{zh_CN-Adobefonts_internal} % Simplified Chinese Support using system fonts
\usepackage{linespacing_fix} % disable extra space before next section
\usepackage{cite}

\begin{document}
\pagenumbering{gobble} % suppress displaying page number

\name{苏畅}

\basicInfo{
  \email{sam.suchang@qq.com} \textperiodcentered\ 
  \phone{(+86) 153-1107-0339} \textperiodcentered\ 
  \faGithub{https://github.com/IcyCC}}
 
\section{\faGraduationCap\  教育背景}
\datedsubsection{\textbf{北京林业大学}, 北京}{2016 -- 至今}
\textit{在读本科生}\ 网络工程专业, 预计 2020 年 9 月毕业

\section{\faUsers\ 实习}
\datedsubsection{\textbf{知乎}}{2018-05-2018.9}
\role{实习后端工程师}{社区事业部-内容}
工作内容:
\begin{itemize}
  \item 配合首页和算法部门, 进行ops的数据标注平台前后端开发
  \item 制作ops平台的代码生成工具, 方便其他业务线快速接入ops平台
  \item 维护配置化爬虫
\end{itemize}

\datedsubsection{\textbf{木瓜移动}}{2018-05-2018.9}
\role{实习后端工程师}{papaya-ads}
工作内容:
\begin{itemize}
  \item 优化广告数据采集脚本,采集速度提升 \textbf{40 \%} 以上
  \item 进行运营后台业务开发, 优化运营后台首页加载速度
  \item 接入dsp的供应方平台接口
\end{itemize}


\section{\faBook\ 近期项目}

\datedsubsection{\textbf{Reimu -- 基于 Reactor 模式的多线程 Morden C++ 网络库}}{\faGithub https://github.com/IcyCC/reimu}

\role{C++, Socket, 并发, io复用}{个人项目}
\begin{onehalfspacing}
项目亮点:
\begin{itemize}
  \item 基于 Poll + Eventloop + 线程池的模式, 实现对 tcp的客户端\(\)服务端 和 定时任务 的并发编程的完善的封装.
  \item 编写多线程同步工具 SafeQueue, 并基于此实现了线程池, 用户的操作会放入线程池中操作以提升并发性能.
  \item 设计了一套Buffer + 带有引用计数的Slice 的机制,减少了字符串的拷贝.
\end{itemize}
\end{onehalfspacing}

\datedsubsection{\textbf{WhParser -- 一个支持类文华财经语法 dsl 的量化交易框架}}{}
\role{Python C++ Cython Ply}{商业项目}
\begin{onehalfspacing}
项目亮点:
\begin{itemize}
  \item 基于 Ply 实现了一个类似文化财经语法的 DSL 及其交易回测运行环境
  \item 基于开源回测框架修改, 简化代码并接入自有数据源, 支持了分钟级别以及夜盘的期货回测.
  \item 处理自有期货数据, 包括合约拆分, 逻辑交易时间计算, 主力合约确定 等
  \item 接入 CTP 接口进行实盘行情和交易的处理
\end{itemize}
\end{onehalfspacing}


\datedsubsection{\textbf{AsyncEasyapi -- 快速后端增删改查API工具, 支持asyncio}}{\faGithub https://github.com/IcyCC/reimu}
\role{Python C++ Cython Ply}{个人项目}
\begin{onehalfspacing}
项目亮点:
\begin{itemize}
  \item 区别于在知乎编写的代码生成工具, 这次选择更高抽象(元类)而不是模板生成代码的思路来简化重复代码
  \item 通过继承几个基类生成业务代码, 实现后台 Api 快速开发, 极大提升了工作效率, 已经在个人多个项目中采用
\end{itemize}
\end{onehalfspacing}


\datedsubsection{\textbf{北京林业大学督导系统v2.0}}{}
\role{Python Vue Redis Kafka Mysql Mongodb, Gevent}{商业项目}
\begin{onehalfspacing}
实验室的项目, 该系统用于学校教师给教师打分, 本次是对旧版系统的升级, 负责工作如下:
\begin{itemize}
  \item 作为项目\textbf{Owner},进行需求梳理,人员分工,架构设计和大部分前端的开发, 完成了整个系统的业务
  \item 抛弃了老版系统的模板渲染的架构, 采用了前后端分离架构, 前端功能更强, 后端更轻量
  \item 抛弃了老版直接数据库写死问卷的做法, 提供了一个基于Json文本的可视化的问卷编辑系统, 方便的增加或修改评价体系
  \item 使用消息队列, 降低后端各个模块(课程, 教师, 评价, 活动, 咨询, 消息等)的耦合度, 提高了性能和可维护性
  \item 设计并实现了一套可能是\textbf{业内优秀}水平的数据报告渲染方案, 该方案已经在多个项目中采用 
\end{itemize}
\end{onehalfspacing}

\section{\faRssSquare 博客和GitHub}
% increase linespacing [parsep=0.5ex]
\begin{itemize}[parsep=0.5ex]
  \item 知乎: https://www.zhihu.com/people/su-chang-60-5/posts
  \item GitHub: https://github.com/IcyCC
\end{itemize}

\section{\faCogs\ 个人技能}
% increase linespacing [parsep=0.5ex]
\begin{itemize}[parsep=0.5ex]
  \item 编程语言: 熟练掌握Python, 了解其C/C++拓展开发的常用方案.较为熟悉 Morden C++, 熟悉 Golang, JavaScript.
  \item Web后端:熟练掌握 MySQL 和 Redis,
  了解 Nginx, Kafka,  Docker, gRpc
  有良好的架构设计意识
  \item Web前端: 熟练掌握Vue 了解React 拥有\textbf{全栈开发}的能力
  \item 计算机网络: 掌握HTTP协议, 了解TCP/IP协议 , Socket编程 和 常见的并发模型 
  \item 计算机科学常识: 熟练掌握常见的算法以及数据结构,
  了解unix操作系统及其环境编程, 
  了解常见的网络安全攻击手段
\end{itemize}

\section{\faInfo\ 自我介绍}
% increase linespacing [parsep=0.5ex]
  高中开始编程, 自学能力强\\
  不仅热爱计算机技术, 更喜欢思考如何利用技术高效优雅地解决问题\\
  开发经验丰富, 大学期间有约\textbf{10w}行左右的业务代码的开发经验, 合作和沟通能力强, 在校期间作为\textbf{负责人}带领由同学组成的团队开发了\textbf{多个}投入生产的项目, 涉及到各种技术栈, 不仅实现了生活费的自给自足, 更培养了承担责任和在团队中扮演合适角色的能力\\  
  好奇心强, 喜欢通过阅读各种项目的源码理解某个机制的运行原理\\
%% Reference
%\newpage
%\bibliographystyle{IEEETran}
%\bibliography{mycite}
\end{document}
